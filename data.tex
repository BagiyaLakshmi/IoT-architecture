%\documentclass[gray]{jmlr} % test grayscale version
%\documentclass[tablecaption=bottom]{jmlr}% journal article
\documentclass[pmlr,twocolumn,10pt]{jmlr} % W&CP article
a
% The following packages will be automatically loaded:
% amsmath, amssymb, natbib, graphicx, url, algorithm2e

%\usepackage{rotating}% for sideways figures and tables
%\usepackage{longtable}% for long tables

% The booktabs package is used by this sample document
% (it provides \toprule, \midrule and \bottomrule).
% Remove the next line if you don't require it.

\usepackage{booktabs}
% The siunitx package is used by this sample document
% to align numbers in a column by their decimal point.
% Remove the next line if you don't require it.
%\usepackage[load-configurations=version-1]{siunitx} % newer version 
\usepackage{siunitx}

% The following command is just for this sample document:
\newcommand{\cs}[1]{\texttt{\char`\\#1}}% remove this in your real article

% The following is to recognise equal contribution for authorship
\newcommand{\equal}[1]{{\hypersetup{linkcolor=black}\thanks{#1}}}

% Define an unnumbered theorem just for this sample document for
% illustrative purposes:
\theorembodyfont{\upshape}
\theoremheaderfont{\scshape}
\theorempostheader{:}
\theoremsep{\newline}
\newtheorem*{note}{Note}

% change the arguments, as appropriate, in the following:
% \jmlrvolume{LEAVE UNSET}
% \jmlryear{2023}
% \jmlrsubmitted{LEAVE UNSET}
% \jmlrpublished{LEAVE UNSET}
% \jmlrworkshop{Conference on Health, Inference, and Learning (CHIL) 2023} % W&CP title

% The optional argument of \title is used in the header
\title[Short Title]{  Protocol for IoT Network Research and Development \titlebreak ref: http://paper.ijcsns.org/07_book/202203/20220307.pdf}

% Anything in the title that should appear in the main title but 
% not in the article's header or the volume's table of
% contents should be placed inside \titletag{}

%\title{Title of the Article\titletag{\thanks{Some footnote}}}


% Use \Name{Author Name} to specify the name.
% If the surname contains spaces, enclose the surname
% in braces, e.g. \Name{John {Smith Jones}} similarly
% if the name has a "von" part, e.g \Name{Jane {de Winter}}.
% If the first letter in the forenames is a diacritic
% enclose the diacritic in braces, e.g. \Name{{\'E}louise Smith}

% \thanks must come after \Name{...} not inside the argument for
% example \Name{John Smith}\nametag{\thanks{A note}} NOT \Name{John
% Smith\thanks{A note}}

% Anything in the name that should appear in the title but not in the 
% article's header or footer or in the volume's
% table of contents should be placed inside \nametag{}

% Two authors with the same address
% \author{%
%  \Name{Author Name1\nametag{\thanks{A note}}} \Email{abc@sample.com}\and
%  \Name{Author Name2} \Email{xyz@sample.com}\\
%  \addr Address
% }

% Three or more authors with the same address:
% \author{%
%  \Name{Author Name1} \Email{an1@sample.com}\\
%  \Name{Author Name2} \Email{an2@sample.com}\\
%  \Name{Author Name3} \Email{an3@sample.com}\\
%  \Name{Author Name4} \Email{an4@sample.com}\\
%  \Name{Author Name5} \Email{an5@sample.com}\\
%  \Name{Author Name6} \Email{an6@sample.com}\\
%  \Name{Author Name7} \Email{an7@sample.com}\\
%  \Name{Author Name8} \Email{an8@sample.com}\\
%  \Name{Author Name9} \Email{an9@sample.com}\\
%  \Name{Author Name10} \Email{an10@sample.com}\\
%  \Name{Author Name11} \Email{an11@sample.com}\\
%  \Name{Author Name12} \Email{an12@sample.com}\\
%  \Name{Author Name13} \Email{an13@sample.com}\\
%  \Name{Author Name14} \Email{an14@sample.com}\\
%  \addr Address
% }

% Authors with different addresses and equal first authors:
\author{%
% \Name{Anonymous First Author 1}\equal{These authors contributed equally} \Email{abc@sample.com}\\
% \addr University X, Country 1
% \AND
% % footnotemark[1] is to refer to the \equal footnote
% \Name{Anonymous First Author 2}\footnotemark[1] \Email{def@sample.com}\\
% \addr University Y, Country 2
% \AND
\Name{Bagiya Lakshmi S} \Email{bagiyalakshmi59@gmail.com}\\
\addr Shiv Nadar University, chennai
}

\begin{document}

\maketitle

\begin{abstract}
It is always 
important to attract new researchers to work on this area and be a 
part of it. The best way to attract researchers to work in any 
research area and have their interest is to give them a clear 
background and roadmap about it. In this way, researchers can 
easily find a deep point to start their research based on their interest. 
This paper presents an overview and roadmap about IoT 
technologies from the most five vital aspects: IoT architecture, 
communication technologies, type of IoT applications, IoT 
applications protocols and IoT challenges. 

\end{abstract}

\section*{Key contributions/ideas from the author}
The authors has given the definition for IoT, How IoT technologies played vital role in simplifying the work.  This paper presents an 
overview and roadmap about IoT technologies from the 
most five vital aspects. Based on these aspects, a researcher 
might be able to create a clear plan to start their research. 
The first aspect is the fundamental of the IoT structure and 
its layers. The second aspect is related to the 
communication technologies that link IoT devices in a 
network. The third aspect is about selecting an environment 
area for deploying IoT devices to provide services to endusers, based on the IoT application types. The fourth aspect 
is related to the communication protocols at the application 
layer. The final aspect concerns the challenges that IoT 
systems face. 

They gave the basic architure diagram of IoT. They say that IoT architecture contains four components namely \textbf{Preception layer, Middleware layer, Application layer and Network layer}.

They have given detailed explanations of various communication technologies.And also IoT applications have been explained breifly. Let's dive into Architecture of IoT.

\section{Background and Roadmap for IoT (IoT 
Architecture)}
\label{sec:intro}

 The IoT architecture can be categorised into four main 
layers: perception layer, network layer, middleware layer 
and application layer.  The main 
objective of this classification is to assist IoT developers in 
identifying the area of any technical issues based on this 
classification. 

\subsection{The Perception Layer }
The perception layer is the first layer that the IoT system 
begins to execute. It is likely same as the physical layer in 
the OSI network. This layer is responsible for collecting and 
exchanging data from the surrounding areas in the physical 
world. Sensors and actuators are the two primary items 
that can detect and sense the changes in the real-world 
environment, for example measuring the temperature of a 
room and then sending the collected data to the next layer, 
which is the network layer for connectivity. 
\textbf{Sensors}: A sensor is an electronic device that can detect and 
sense the physical environment such as measuring the 
temperature by using a thermistor.
\textbf{Actuators}: An actuator is a machine that can move the IoT 
devices from one state to another state such as switching the 
light off or on by using a Rely device . 
 \textbf{IoT Data}: The IoT objects such as sensors and actuators 
generate two types of data. Measurement data is when 
sensors generate data to detect and sense the events in real 
world of the surrounding environment such as temperature, 
humidity etc. Context-data provides information about the 
description of an object and its condition such as battery life, latency, etc. and sends this information to the users. 
 IEEE 802.15.4 was introduced in 2003 as a wireless 
personal area network standard for the physical layer and 
MAC, which is preferable in cases where high power and 
high-rate wireless communication systems are not required 
. It covers small areas only, and the maximum 
transmission range that it can reach is about 100m


\subsection{The Network Layer }
The network layer is the second layer in the IoT 
architecture. It acts as the brain of the IoT systems [11]. The 
primary aim of the network layer is to gather data from the 
perception layer and transmit this information to the 
middleware layer for further analysis and processing. 
Internet getaways such as WiFi, RFID, etc. operate at this 
layer to execute different network communication services . In this section, the communication technologies, 
routing, and the architecture of IoT networks will be 
discussed. 
\newline
\textbf{Communication technologies} can be defined as the 
mechanism type that links IoT devices in a network for the 
purpose of data transmission. There are various 
communication mechanisms in the market, such as WiFi. In 
this section, seven communication mechanisms with their 
description are listed below. There are more than these 
seven in the market; however, we believe the following 
seven are the most important.
\newline
\textbf{Sigfox}: Sigfox could be considered as the first global IoT 
network in which IoT devices can transmit data without the 
need to install any network connections. The management of 
transmitting data between IoT devices proceeds in the cloud 
by the software-based communication that the Sigfox offers. 
This management leads to minimising the cost of 
connectivity and power consumption [13]. The Sigfox 
network architecture consists of three majors’ parts: base 
stations, IoT devices and central networks. The 
communication protocol in Sigfox is designed to send small 
messages (0 to 12 bytes).
\newline
\textbf{NB-IOT}: Researchers are giving more concern to the NBIoT due to its low-cost, low power consumption, longdistance indoor coverage. It is the most popular choice for 
most of the IoT nodes [15]. The bandwidth of NB-IoT for 
both uploading and downloading is the best choice for lowcost devices, and it is about 180 kHz, which is considered as 
a low-frequency bandwidth. NB-IoT provides connectivity 
for IoT devices over long-distances, and the maximum 
coverage that it can serve is about 15km. The latency in NBIoT is preferable in many IoT applications, which is about 
10ms.
\newline 
\textbf{Zigbee}: Zigbee has been launched as a wireless 
communication protocol. In comparison with other 
communication protocols, the cost of establishing a ZigBee 
network is low. It covers small areas with a low data rate and 
can provide service monitoring in small areas, such as homes 
. The coverage of a Zigbee network is the same as a WiFi network because both provide the same bandwidth, which 
is 2.4 GHz. 
\newline 
\textbf{NFC}: Near Field Communication was introduced as 
wireless communication technology to provide connectivity 
in a very small area. NFC has added value and brings many 
advantages to the IoT technology. One of its remarkable 
gains is in the way of communications, since it does not 
require any pairing to set up, so it is much easier than 
Bluetooth, which requires paring . The main 
disadvantage of NFC is the short coverage, which is about 4 
cm . 
\newline 
\textbf{Radio Frequency Identification}: RFID can be classified 
as one of the wireless communication technologies. The 
primary objective of RFID is to collect data from the 
surrounding areas in a limited range from 1m to 12m by 
broadcasting radio signals and it processes this information 
to implement services such as monitoring, tracking, etc. 
RFID tags and RFID readers are the two main parts of the 
RFID system; tags such as cards collect information by 
broadcasting radio waves, whereas readers act as a brain to 
execute processing. 
\newline 
\textbf{Bluetooth}: Bluetooth technology can be utilised to provide 
connectivity for IoT devices in a limited range. The 
maximum range that Bluetooth can cover is about 10m.

\subsection{The Middleware Layer }
The middleware layer is the third layer in the IoT 
architecture. It analyzes and stores the received data from the 
network layer. It works as a bridge to link the IoT 
system to the computing systems and databases for further 
processing. It is responsible for preparing the data to be 
utilized in the application layer. Machine learning and 
artificial intelligence systems might be used at this layer to 
transform the collected data into valuable information to 
support the system in the decision-making process at the 
application layer. 
\subsection

\subsection{The Application Layer }
The application layer is the top layer in the IoT 
architecture and its main purpose to provide services for the 
end-users . This layer generates the processed data from 
the middleware layer to meet the QoS in IoT applications in 
various cases to deliver services to the end-user. It works as 
a chain to enable end-users to use IoT applications such as 
smart homes, smart cities, smart industries, etc. End-user can 
access these IoT applications through internet-enabled 
devices such as smartphone, laptop, television, etc.

\newline 
\textbf{IoT in healthcare}: The main objective of proposing IoT is 
to provide more convenient life for humans by organizing 
their basic tasks. The treatment of patients in the healthcare 
systems can be enhanced when the IoT devices utilized in 
the system [34]. This can be achieved by combining the IoT 
sensors with the health monitoring gadgets used by patients 
to provide further analyses. These sensors gather 
information about the status of patients and send this 
information to the internet for further processing. Doctors 
and nurses use the analysed and processed data to monitor 
the status of patients remotely [35]. Moreover, healthcare 
systems might witness significant improvements when 
applying IoT systems in the environment regarding data 
accuracy when reporting the patient's status like their 
temperature and blood pressure to doctors and nurses 
compared to writing the data manually where mistakes 
might happen. 
\newline 
\textbf{IoT in smart buildings}: The smart building has been 
established to provide more convenient living arrangements 
for residents. The IoT systems can monitor and control the 
appliances in building like remote monitoring via Internet 
[36]. Switching appliances off and on remotely through 
smartphone apps can play an important role in reducing the 
power consumption as the control of these appliances can 
be done easily. IoT systems can also provide safety 
monitoring to protect residents from any external risks by 
using cameras and alarm systems effectively [37]. In 
addition, smart buildings might play an important role not 
directly in decreasing the conflicts between the family 
members because everything in the house can be handled 
easily. For instance, an intelligent vacuum can clean the 
living room automatically, which will allow the family 
members to have more time to set together in a clean place 
without making any effort. 
\newline 
\textbf{IoT in transportation}: Nowadays, the demand for the 
public transportation system (PTS) has risen recently due to 
the significant increase in the number of daily trips because 
of growing urbanization. The demand for PTS in urban 
cities is high, and the traditional PTS has achieved 
significant contributions in terms of reducing air pollution, 
traffic accidents and road congestion [38]. One of the main 
disadvantages of using public transportation is the time that 
passengers spend in stations waiting for buses and trains. 
Knowing the location of buses and the exact arrival time 
will encourage commuters to use public transportation in 
their daily life, and this can be achieved by integrating the 
IoT systems into PTS [39]. From an economic perspective, 
the income of public transportation might increase when 
integrating IoT systems in their environment because the 
number of passengers will increase when the trips are 
scheduled accurately. 
\newline 
\textbf{IoT in smart energy}: The IoT sensors can be integrated 
with electronic gadgets to measure and analyse the power 
consumption of these gadgets for further processing and 
monitoring [40]. Monitoring power consumption 
effectively can play an essential role in reducing the cost of 
bills. It also benefits to the environment in several ways 
such as minimizing the air pollution. Electric companies 
will also benefit, as this monitoring will decrease the 
pressure and load on these companies in terms of reading 
and reporting consumers' bills [41]. All that will lead to 
increase the confidence between the electric companies 
IoT in agriculture: Establishing the IoT technology in the 
agriculture environment can play an essential role in 
improving the farming environment. Farmers can easily 
monitor their crop yield when effectively using IoT devices 
[42]. This technology will bring several benefits to the 
agriculture environment in term of many aspects. It helps 
farmers to limit the time to produce more with less effort. 
The performance of the production can be examined after 
generating the data from the IoT sensors [43]. The IoT 
sensors will assist in increasing the success of crop 
production as these sensors can make destinations in early 
stages; for instance, greenhouse agriculture can be closed 
directly when sensors detect heavy rain. 
\newline 
\textbf{IoT in industry}: IoT systems in industry can handle and 
control the manufacturing processes at the real-time without 
any delays. The M2M communications can play an essential 
role in reducing the number of workers in industries, which 
minimize the cost of manufacturing. Integrating IoT 
systems into various industries has enhanced efficiency, 
improved the QoS as well as maintenance services. 
Supervisors in industries can easily examine the 
performance of manufacturing by pulling the data from the 
IoT sensors [44]. From an economic perspective, the 
number of industries in the country might increase as traders 
can establish new industries with less cost compared to the 
past, as the number of workers will decrease when 
integrating IoT systems into their environment.


\section{IoT Challenges}
Researchers are continuously working to overcome the 
challenges faced by IoT technology.
\subsection{Security }
 Securing IoT systems is one of the most critical 
challenges that experts and researchers have been trying to 
overcome. Exchanging data among IoT nodes should be 
protected from any external attack to create a confidential 
IoT environment . Protecting IoT devices will 
encourage consumers to utilise the IoT systems in their 
homes, since they know their privacy is protected


\subsection{Limited power  }
Due to the massive rise in the number of 
IoT devices, there is now a storage demand to promote IoT 
networks' energy efficiency. In addition to the number of 
IoT devices, researchers have proposed several complex 
algorithms to improve other parameters such as security, 
data rate and bandwidth, which can contribute to increasing 
the workload and consuming more power . To the best 
of our knowledge, researchers have to focus on three 
network areas to reduce the power consumption in the IoT 
devices. These areas are the routing and clustering in the 
WSNs, and the fog computing services in the IoT networks.


\subsection{ Availability }
As the IoT devices require real-time processing, the devices 
should be available all the time to avoid delays. In some 
cases, a service might not be available for a user when 
he/she sends a request, which might negatively impact the 
work process, and leading to poor outcomes . Several 
reasons might affect the IoT systems and make it 
unavailable and not ready for use. These reasons can be like 
sensors ageing, dead battery, and more . Researchers 
have to be aware of the quality and the lifetime of the 
sensors and hardware devices to make the system available 
as long as they can,

\subsection{Scalability }
Scalability is the ability of the IoT systems to handle the 
growth in the number of IoT devices while overcoming the 
challenges associated with that growth . Horizontal 
sensing and vertical sensing are the two main features that 
have to be enhanced to improve IoT systems' scalability. 
Adding more devices and nodes to the network belongs to 
the horizontal sensing, whereas increasing the capacity of a 
network by adding more resources such as CBU, RAM, and 
Power belongs to vertical sensing . Achieving a high 
level of scalability will increase the IoT devices' chance to 
become successful in the future and to be more reliable. 
Researchers must focus on finding a way to add more 
resources, such as CPU and IoT devices without affecting 
the system's interoperability. A system is called 
interoperable when multiple IoT sensors can work together 
in the right place and time to perform functions without any 
issues.

\subsection{ Maintenance }

The cost of maintaining IoT devices might be more 
expensive than establishing them. It is vital in IoT to have a 
resiliency system in which nodes can recover and fix errors 
without any interactions to reduce the cost of maintenance. 
There are various ways to reduce the cost of maintenance in 
the IoT environment. The deployment of the IoT devices 
has to be set in such a way that they can easily be located 
when they need to be fixed. It is vital to be aware of the IoT 
devices' range as these devices work with a limited range to 
avoid damages and incorrect results. The quality of the 
IoT sensors has to be high to avoid maintains as much as 
possible.


\section{Conclusion}

This paper aims to attract more new researchers and developers 
to start their research in IoT networks. This paper has 
provided researchers with a comprehensive background 
about the area in terms of five main aspects: IoT 
architecture, communication technologies, type of IoT 
applications, IoT applications protocols and IoT challenges. 
Future studies will investigate and analyse IoT networks' 
five most vital challenges individually to help researchers 
determine their paths when targeting a challenge from the 
five mentioned challenges. 

\end{document}
